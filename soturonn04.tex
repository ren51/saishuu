% !TEX encoding = UTF-8 Unicode

\documentclass[twocolumn,10pt,a4j]{jsarticle}
\usepackage{kougai}

\title{論理的文章のアウトラインの作成を支援するツールの開発}
\author{1632144 三浦 恋  指導教員 須田 宇宙 准教授}
\date{}

\begin{document}

\maketitle

% 1章
\section{はじめに}

%背景
大学生に対して,論理的な思考力や論理的文章作成能力の要求が高まっている.
しかし,論文やレポートを書く際にアウトラインなどの事前準備をせずに文章の作成を行ってしまう学生が多く,論理的な文章にならないことが問題点として挙げられる.
そのため,レポートの書き方の指導や修正を行うライティングセンターの設置などが進められているが
自発的に利用しなければ文章作成力は向上しない.

%問題点
一般に論文や小説などの長文を作成するためのツールとして,アウトラインプロセッサが使用されることが多い.
これは,文章を階層的に管理することに主眼が置かれており,学生にとって主張や根拠などが明確な一貫した文章を書く力を養うためのツールではないことが問題点となっている.

%目的
そこで本研究では,主張や根拠などが明確な一貫した文章を書く力を身に付けるための論文やレポートのアウトラインの作成を支援するアカデミックアウトラインツールを開発することを目的としている.

% 2章
\section{アカデミックライティングについて}
% 既存のツールで不足する点
大学では答えのない問題を扱い,問題に対して自分の考えを主張することが必要とされている.そこで,論文やレポート等には下記(1)〜(5)が求められる.このような文章を書く技術,書く行為はアカデミックライティングと呼ばれている\cite{ren01}.

% アカデミックライティングツールの理想
\begin{description}
  \item[(1)] 主張と根拠が明示されている
  \item[(2)] 問いと答えの構造と論理的な説明での構成されている
  \item[(3)] 引用の倫理のルールに従っている
  \item[(4)] パラグラフ構造になっている
  \item[(5)] 学術的文章に特有の一定の形式に従ってる
\end{description}

% 3章
\section{開発したツールについて}
% アカデミックライティングとコンセプトについてを書く
本制作では論理的な構成の文章を書く際の準段階のために,主張と根拠の確認や参考文献の管理を行うことができるツールが必要だと考えた.また,通学時間などの隙間時間で,意見や構成の整理を行えることを目指した.

% 2章から導かれた,求められる機能
主張が一貫した論理的文章のアウトラインを作成するために,2章で述べた(1)〜(3)の考えの整理の補助,文章構成の整理,スマートフォンでの利用を考えたツールを開発した.(4),(5)においては文章に起こした際の特徴であるため,本制作では対象外とした.また実際に書く文章は別のアプリケーションで記述することにする.

本ツールの具体的な機能を\UTF{2460}〜\UTF{2463}で記載し,動作例と機能\UTF{2460}〜\UTF{2463}の対応したものを図\ref{fig:g}に示す.
\begin{description}
  \item[\UTF{2460}]主張と根拠の明確化
  
 見返した際に主張からずれた意見が出ることを防ぐため,主張と根拠を登録する機能
  
  \item[\UTF{2461}]課題に対する疑問とその答えの記入
  
課題に対する疑問とそれに対する答えを記述する機能
  
  \item[\UTF{2462}]論理的な構成の整理
  
  一般的なアウトラインプロセッサと同様に論理的な文章を書く上で各内容の順番や情報を整理するため順番を入れ替える機能,章や段落の情報を表示する機能
  
  \item[\UTF{2463}]参考文献の管理
  
  文章を作成する際に引用した文献を確認,整理する機能 
  
 % \item[\UTF{2464}]PWA(Progressive Web Apps)
  
  %モバイル端末でページを表示する際にネイティブアプリと同様の挙動をさせることができる技術
\end{description}

\begin{figure}[h]
\begin{center}
 \includegraphics[clip,width=85mm,height=55mm]{pp01.pdf}
\end{center}
 \caption{実際の画面}
 \label{fig:g}
\end{figure}

本ツールは作業環境として自宅や学校のみならず隙間時間に利用することを視野に入れ,PCとスマートフォンの両方からアクセスでの利用を考えた.そこでWeb上で動作するツールが望ましいと考え,HTML5,CSS3, JavaScriptを使用し開発を行った.また,モバイル端末でページを表示する際にネイティブアプリと同様の挙動をさせることができる技術であるPWA(Progressive Web Apps)を実装した.
% 4章
\section{おわりに}
本制作では主張と根拠の確認や参考文献の管理をすることで,主張の一貫した文章を作成できると考え,論理的文章のアウトライン作成を支援するツールの開発を行った.

\begin{thebibliography}{99}
\bibitem{ren01} 堀 一成,坂尻 彰宏:"阪大生のためのアカデミックライティング",
\url{https://ir.library.osaka-u.ac.jp/repo/ouka/all/27153/Academic%20Writing%20Introduction.pdf}, 2019/8/23参照


\end{thebibliography}

\end{document}
