% !TEX encoding = UTF-8 Unicode

\documentclass[twocolumn,10pt,a4j]{jsarticle}
\usepackage{kougai}

\title{論理的文章のアウトラインの作成を支援するツールの開発}
\author{1632144 三浦 恋  指導教員 須田 宇宙 准教授}
\date{}

\begin{document}

\maketitle

% 1章
\section{はじめに}

%背景
大学生に対して,論理的な思考力や論理的文章作成能力の要求が高まっている.
しかし,論文やレポートを書く際にアウトラインなどの事前準備をせずに文章の作成を行ってしまう学生が多く,論理的な文章にならないことが問題点として挙げられる.
そのため,レポートの書き方の指導や修正を行うライティングセンターの設置などが進められているが
自発的に利用しなければ文章作成力は向上しない.

%問題点
一般に論文や小説などの長文を作成するためのツールとして,アウトラインプロセッサが使用されることが多い.
これは,文章を階層的に管理することに主眼が置かれており,学生にとって主張や根拠などが明確な一貫した文章を書く力を養うためのツールではないことが問題点となっている.

%目的
そこで本研究では,主張や根拠などが明確な一貫した文章を書く力を身に付け,論文やレポートの作成を支援するアカデミックアウトラインツールを開発することを目的としている.

% 2章
\section{アカデミックライティングについて}
% 既存のツールで不足する点
一般に利用されているアウトラインプロセッサは全体の構造を組み立て,見出しをつけていき,文章を階層的に管理することのできるソフトウェアのことを指す.
しかし大学で学生に作成が求められる論文やレポート等には以下(1)~(5)の特徴が挙げられる.このような文章を書く技術,書く行為,または書いた物のことをアカデミックライティングと呼ぶ\cite{ren01}.

% アカデミックライティングツールの理想
\begin{description}
  \item[(1)] 主張と根拠が明示されている
  \item[(2)] 問いと答えの構造と論理的な説明での構成されている
  \item[(3)] パラグラフ構造になっている
  \item[(4)] 引用の倫理のルールに従っている
  \item[(5)] 学術的文章に特有の一定の形式に従ってる
\end{description}

% 3章
\section{開発したツールについて}
% 2章から導かれた,求められる機能
本制作では主張が一貫した論理的文章のアウトラインを作成するために,2章で述べた(1),(2),(4)を補助し,文章の構成を整理するツールを開発した.(3),(5)においては文章に起こした際の特徴であるため,本制作では対象外としている.また実際に書く文章は別のアプリケーションで記述することにする.

% アカデミックライティングとコンセプトについてを書く
一般的なアウトラインプロセッサと異なり,主張と根拠の明確化や,参考文献を管理することに主眼を置き,アカデミックライティングに沿った文章を作成するために以下の機能を持たせた.

\begin{description}
  \item[\UTF{2460}]主張と根拠の明確化
  
  見返した際に主張からずれた意見が出ることを防ぐため,主張と根拠を登録する機能
  
  \item[\UTF{2461}]課題に対する疑問とその答えの記入
  
 課題に対する疑問とそれに対する答えを記述する機能
  
  \item[\UTF{2462}]論理的な構成の整理
  
  一般的なアウトラインプロセッサと同様に論理的な文章を書く上で各内容の順番や情報を整理するため順番を入れ替える機能,章や段落の情報を表示する機能
  
  \item[\UTF{2463}]参考文献の管理
  
  文章を作成する際に引用した文献を確認,整理する機能 
\end{description}

%コメントアウトしてます
\begin{comment}
実装した機能として\textcircled{\scriptsize{1}}は「主張と根拠の対応」,\textcircled{\scriptsize{2}}は「課題に対する疑問とその答えの記入」,\textcircled{\scriptsize{3}}は「論理的な構成の整理」,\textcircled{\scriptsize{4}}「参考文献の管理」とする.
\textcircled{\scriptsize{1}}「主張と根拠」ではまず主張の表示をした.
また根拠は複数存在する可能性があるため根拠の追加も行えるようにした.
\textcircled{\scriptsize{2}}「問いと答え」では課題や主張への疑問とそれに対する答えを短い文章で記述し,見返した際に主張からずれた意見が出ることを防ぐことができると考え,課題や主張への疑問と答えを短い文章で記述する.
\textcircled{\scriptsize{3}}「論理的な構成」では論理的な文章を書く上で各内容の順番や序論,本論,結論のどの部分の情報なのかを整理するため順番を入れ替える機能,文章がどの章に情報なのか表示する機能にした.
\textcircled{\scriptsize{4}}「参考文献」では文章を作成する際の引用した文献を整理することが目的であり,どの文献が何章で参照を行ったのか確認できるように参考文献にはラベルを付けた.
\end{comment}
%ここまでコメントアウトしてます

\begin{figure}[h]
\begin{center}
 \includegraphics[clip,width=85mm,height=55mm]{pp09.pdf}
\end{center}
 \caption{ツールの構成}
 \label{fig:教科書}
\end{figure}

% 4章
\section{おわりに}
本制作では論理的文章のアウトライン作成を支援するツールの制作を行った.

\begin{thebibliography}{99}
\bibitem{ren01} 堀 一成,坂尻 彰宏:"阪大生のためのアカデミックライティング",
\url{https://ir.library.osaka-u.ac.jp/repo/ouka/all/27153/Academic%20Writing%20Introduction.pdf}, 2019/8/23参照


\end{thebibliography}

\end{document}
